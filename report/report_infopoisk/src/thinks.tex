\section{Заключение}

\subsection{Итоги проекта}

В ходе выполнения проекта была разработана полнофункциональная поисковая система для Fashion Corpus, включающая все этапы обработки информации:

\begin{enumerate}
    \item \textbf{Сбор данных:} Создан робот, собравший 30,166 документов из Википедии и модных изданий
    \item \textbf{Предобработка:} Реализована токенизация со скоростью 2.4 МБ/сек и точностью 94.9\%
    \item \textbf{Анализ:} Проведен анализ распределения Ципфа (R² = 0.983) с учетом специфики модной терминологии
    \item \textbf{Стемминг:} Внедрен стеммер, сокративший словарь на 28.4\% при улучшении полноты поиска на 15.6\%
    \item \textbf{Индексация:} Разработан бинарный формат индекса с коэффициентом сжатия 0.792
    \item \textbf{Поиск:} Реализован булев поиск со временем ответа 2.8 мс и точностью 72\%
\end{umerate}

\subsection{Достигнутые результаты}

\begin{table}[H]
\centering
\caption{Ключевые достижения проекта}
\begin{tabular}{|l|r|l|}
\hline
\textbf{Показатель} & \textbf{Значение} & \textbf{Комментарий} \\
\hline
Размер корпуса & 30,166 документов & Репрезентативная выборка \\
Скорость токенизации & 2,401 КБ/сек & Выше аналогов \\
Качество стемминга & F1=87.6\% & Сопоставимо с Porter2 \\
Сжатие индекса & 79.2\% & Эффективный формат \\
Время поиска & 2.8 мс & Быстрый отклик \\
Точность поиска & Precision@10=72\% & Конкурентоспособно \\
\hline
\end{tabular}
\end{table}

\subsection{Научная и практическая значимость}

\textbf{Научная значимость:}
\begin{itemize}
    \item Исследование распределения Ципфа для специализированных корпусов
    \item Анализ эффективности стемминга для модной терминологии
    \item Разработка оптимизированных алгоритмов для булева поиска
\end{itemize}

\textbf{Практическая значимость:}
\begin{itemize}
    \item Готовая система для поиска по модным текстам
    \item Модульная архитектура для повторного использования
    \item Пример реализации всех этапов поисковой системы
\end{itemize}

\subsection{Ограничения и направления развития}

\subsubsection{Текущие ограничения}

\begin{enumerate}
    \item \textbf{Масштабируемость:} Работает с корпусами до 2 ГБ в памяти
    \item \textbf{Многоязычность:} Только английский язык
    \item \textbf{Типы поиска:} Только булев поиск, нет ранжирования
    \item \textbf{Формат документов:} Поддерживает только текст
\end{enumerate}

\subsubsection{Перспективы развития}

\begin{enumerate}
    \item \textbf{Поддержка векторного поиска:} Внедрение эмбеддингов и семантического поиска
    \item \textbf{Ранжирование:} Реализация BM25 или нейросетевых моделей
    \item \textbf{Мультиязычность:} Добавление поддержки других языков
    \item \textbf{Распределенная обработка:} Поддержка кластерных конфигураций
    \item \textbf{Обучение с подкреплением:} Адаптация к поведению пользователей
\end{enumerate}

\subsection{Рекомендации для практического применения}

\subsubsection{Для исследователей моды}

\begin{itemize}
    \item Использовать для анализа исторических тенденций
    \item Применять для поиска информации о конкретных дизайнерах или техниках
    \item Использовать как инструмент для сравнительного анализа текстов
\end{itemize}

\subsubsection{Для образовательных учреждений}

\begin{itemize}
    \item Пример для курсов по информационному поиску
    \item Демонстрация полного цикла обработки информации
    \item Базис для студенческих проектов и исследований
\end{itemize}

\subsubsection{Для индустрии}

\begin{itemize}
    \item Основа для коммерческих поисковых решений в моде
    \item Интеграция с системами управления контентом
    \item Использование в аналитических платформах для модного бизнеса
\end{itemize}

\subsection{Заключительные замечания}

Разработанная поисковая система для Fashion Corpus представляет собой законченное решение, демонстрирующее все этапы создания поискового движка: от сбора данных до реализации сложных запросов. Система сочетает академическую строгость с практической применимостью, предлагая эффективные алгоритмы для работы со специализированными текстами.

Особенностью проекта является учет специфики модной терминологии, что делает систему более релевантной для целевой предметной области по сравнению с общими поисковыми решениями.

Проект может служить как образовательным ресурсом для изучения информационного поиска, так и основой для коммерческих приложений в модной индустрии.

\end{document}
 
\pagebreak